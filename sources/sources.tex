\begin{figure}[h!]
    \includeSource[4in]{GdP-Miraba}
    \def\GdPMirabaCaption{%
        Gutiérrez de Padilla, \wtitle{Miraba el sol el águila bella},
        \signature{MEX-Pc}{Leg. 34}, sole extant Tenor part
        (photograph by the editor, courtesy of the Capitular Archive of Puebla
        Cathedral)%
    }
    \caption[\GdPMirabaCaption]{\GdPMirabaCaption\footnotemark}
\end{figure}

\footnotetext{%
    The inclusion of these images is deemed to constitute fair use. 
    The original sources are in the public domain, and the images are edited
    and presented in an interpretive context, in a scholarly edition whose
    license proscribes commercial use.%
}

\newpage
\begin{figure}[ht!]
    \includeSource{GdP-jacara}
    \caption{%
        Gutiérrez de Padilla, \wtitle{Afuera, afuera pastores (Jácara)},
        Tenor I performing part,
        \signature{MEX-Pc}{Leg. 1/3}, \wtitle{Navidad de el año de 1652}
        (image from microfilm, courtesy of the Capitular Archive of Puebla
        Cathedral)%
    }
\end{figure}

\begin{figure}[hb!]
    \includeSource[3in]{GdP-jacara-poem}
    \caption{%
        \wtitle{Afuera, afuera pastores (Romance a lo valentón)}, source for
        \term{jácara} by Gutiérrez de Padilla, in poetry imprint by Feliz
        Persio Bertiso, \wtitle{La Harpa de Belén} (Seville, 1634), in
        binder's collection preserved in Puebla,
        \signature{MEX-Plf}{80070-42010404}
        (image courtesy Biblioteca José María Lafragua, Benemérita Universidad
        Autónoma de Puebla)%
    }
\end{figure}

\newpage
\begin{figure}[ht!]
    \includeSource{Bruna-Suban}
    \caption{%
        Bruna, \wtitle{Suban las voces}, Tiple 1 part from previously
        unattributed set of manuscript parts, \signature{E-Bbc}{M759/44}
        (image courtesy Biblioteca de Catalunya)%
    }
\end{figure}

\begin{figure}[hb!]
    \includeSource{Ambiela-Suban}
    \caption{%
        Ambiela, \wtitle{Suban las voces}, Tiple I-1 part,
        \signature{E-Bbc}{M733/1} (image courtesy Biblioteca de Catalunya)%
    }
\end{figure}

\newpage
\begin{figure}[ht!]
    \includeSource{Irizar-full}
    \caption{%
        Irízar, \wtitle{Qué música celestial}, manuscript score
        draft of opening written in notebook made from received letters,
        \signature{E-SE}{18/36} (photograph by the editor, courtesy Segovia
        Cathedral Archive)%
    }
\end{figure}

\begin{figure}[hb!]
    \includeSource{Irizar-detail}
    \caption{Irízar, \wtitle{Qué música celestial}, score draft, detail}
\end{figure}

\newpage
\begin{figure}[ht!]
    \includeSource{Irizar-poem}
    \caption{%
        \wtitle{Qué música celestial}, poem later attributed to Manuel de León
        Marchante, in imprint from Christmas 1677 at Toledo Cathedral, source
        for Irízar's 1678 setting, \signature{E-Mn}{R/34982}
        (image from microfilm, courtesy Biblioteca Nacional de España)%
    }
\end{figure}

\begin{figure}[hb!]
    \includeSource{Carrion-SIi}
    \caption{%
        Carrión, \wtitle{Qué destemplada armonía}, Tiple I-1 part, 
        \signature{E-SE}{20/5}
        (photograph by the editor, courtesy Segovia Cathedral Archive)%
    }
\end{figure}

\endinput
