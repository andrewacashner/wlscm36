% MIRABA EL SOL EL AGUILA BELLA
% Set by Juan Gutierrez de Padilla in MS partbook collection, MEX-Pc
% De concepción
\begin{poemtitleblock}
    \poemhead{\wtitle{Miraba el sol el águila bella} (Puebla, before 1660)}
    \poemsource{%
        Anonymous, from musical setting by Juan Gutiérrez de Padilla, in 
        \wtitle{Villanços dibinos, i umanos de diversos autores}
        (\signature{MEX-Pc}{Leg. 34})%
    }
\end{poemtitleblock}

\begin{poemtranslation}
    \begin{original}
        \StanzaSection{4}[\add{Introducción}]
        Miraba el sol &
        el águila bella, &
        y viéndola yo &
        su limpio crisol,
        \SectionBreak

        \StanzaSection{6}[\add{Estribillo}]
        en la sol-fa mire, &
        pues agraciada en un punto, &
        en un punto se ve &
        con tanto re-mi-fa-sol,	&
        que sola a la luz de mi sol &
        la admire de mi sol. \&

        \Stanza{5}
        Ave, mirasol, Ave, &
        y la gracia fue &
        que el milano nunca la mire, &
        y ella al sol mire,	&
        y la mire el sol.
        \SectionBreak

        \StanzaSection{6}[Responsión]
        Ave, mirasol, Ave, &
        Ave mi mirasol, Ave, &
        y la gracia fue	&
        que el milano nunca la mire, &
        y ella al sol mire, &
        y la mire el sol.
        \SectionBreak

        \StanzaSection{8}[Coplas]
        1. A el cielo su canto alegra &
        pues que el sol la acrisola. &
        Ni una se mínima sola &
        tuvo de \critnote{la nota negra}
        {A blackened notehead, using mensural coloration to indicate an
        imperfect note value and usually a syncopated rhythmic pattern}, &
        cantaba en dulce bemol &
        sin que Adán tono le de	&
        \emph{su limpio crisol}	&
        \emph{en la sol-fa mire,} 
        \SectionBreak[\add{Estribillo rep.}]

        \Stanza{8}
        2. Al nido y punto de voz &
        entró de instantes sin tedio, &
        que no hay tiempo de por medio &
        en la máxima de Dios, &
        sobre un errante farol,	&
        eco el compás con el pie. &
        \emph{su limpio crisol}	&
        \emph{en la sol-fa mire,}
        \SectionBreak[\add{Estribillo rep.}]
    \end{original}

    \begin{translation}
        \StanzaSection{4}
        The beautiful eagle &
        was regarding the sun, &
        and I, seeing her \add{the eagle}, &
        would regard her pure crucible
        \SectionBreak

        \StanzaSection{6}
        in solfa, &
        since, graced in one point \add{of time}, &
        she is seen as a point/note, &
        with so much re-mi-fa-sol, &
        that only in the light of my sun, &
        should she be admired by my sun.
        \SectionBreak

        \StanzaSection{5}
        Ave/Hail, sunflower, hail--- &
        and the grace was &
        at the kite should never regard her &
        while she should regard the sun, &
        and the sun regard her.
        \SectionBreak

        \StanzaSection{6}
        Ave/Hail, sunflower, hail, &
        Hail, my sunflower, hail, &
        and the grace was &
        at the kite should never regard her &
        while she should regard the sun, &
        and the sun regard her.
        \SectionBreak

        \StanzaSection{8}
        Her song rejoices up to the sky, &
        for indeed the sun purifies her. &
        She alone took not even a minim &
        from the black note, &
        she sang in a sweet flat &
        without taking her pitch/mode from Adam. &
        \emph{Regard her pure crucible} &
        \emph{in solfa,}
        \SectionBreak

        \Stanza{8}
        Into the nest on the point, with strong voice &
        she entered in an instant, without tedium, &
        for there is no half tempo &
        in the maxima of God, &
        upon an errant lantern, &
        echo the meter with the foot. &
        \emph{Regard her pure crucible} &
        \emph{in solfa,} \&
    \end{translation}
\end{poemtranslation} 
\endinput
