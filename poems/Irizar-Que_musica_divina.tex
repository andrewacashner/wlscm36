% Irízar Qué música celestial poem
\begin{poemtitleblock}
    \poemhead{\wtitle{Qué música celestial} (Segovia, 1678)}
    \poemsource{%
        Attr. Manuel de León Marchante, \wtitle{Obras poeticas posthumas}
        (Madrid, 1733), 169--170 (\signature{E-Mn}{R/34982, VE/88/42})%
    }
\end{poemtitleblock}

\begin{poemtranslation}
    \begin{original}
        \StanzaSection{4}[Estribillo]
        Qué música celestial &
        es la que hoy el aire altera? &
        Qué soberana armonía &
        es la que el oído eleva? \&

        \Stanza{4}
        Qué luz es esta que en día &
        transforma la noche densa? &
        Qué claro fulgor el cielo &
        esta noche da a la tierra? \&

        \Stanza{2}
        Lo admirable de este Enigma &
        grande novedad encierra. \&

        \Stanza{4}
        Gloria repiten las vozes. &
        Paz dan sus luces cadencias. &
        Toda la tierra es ya Cielo &
        Y todo el Cielo da en tierra. \&

        \Stanza{4}
        Qué será que en nuestra duda &
        no cabe saber qué encierra &
        ser el cielo voces todo, &
        ser glorias toda la tierra?
        \SectionBreak

        \Stanza{6}
        \emph{Ángel:} La causa es, Pastores, &
        que de una Doncella &
        forma el Verbo carne, &
        por pagar la deuda, &
        que del primer Padre &
        tomó por su cuenta: \&

        \Stanza{6}
        y el cielo envidioso &
        de ver que hoy la tierra &
        a Dios goza humano, &
        con vozes celebra &
        la dicha que el Orbe &
        posee en su esfera.
        \SectionBreak

        \Stanza{1}
        Venga enhorabuena.
        \SectionBreak

        \Stanza{3}
        Pues si viene a darnos &
        gloria en vez de pena, &
        venga enhorabuena. \&

        \Stanza{4}
        Y de vuestras dudas &
        proseguid el tema, &
        pues en la sustancia  &
        yo diré qué encierran. \&

        \Stanza{4}
        Son admiraciones &
        de ver que en la tierra &
        se siembre luceros, &
        se cojan estrellas. \&

        \Stanza{4}
        Proseguid las dudas, &
        y en graves cadencias &
        de sonoras voces &
        descifraré el tema.
        \SectionBreak

        \Stanza{1}
        Venga enhorabuena. 
        \SectionBreak

        \StanzaSection{4}[Coplas \add{Romance con siguidillas}]
        Que será que en vozes graves  &
        toda la Corte celeste &
        con gloria, y paz nos combida &
        en alternados Motetes? \&

        \Stanza{4}
        Es que el cielo hoy gozoso &
        (de glorias tales) &
        varias galas de acentos &
        rompe en el aire. \&

        \Stanza{4}
        Que será que a media noche &
        por las puertas del Oriente, &
        sin romper el Alba bella, &
        el Sol se nos manifieste? \&

        \Stanza{4}
        Es que el Sol cuando nace &
        de tal Aurora, &
        dora el yerro del hombre &
        que a ella la adora. \&

        \Stanza{4}
        Que será que un Portalillo &
        tanto en si se desvanece, &
        que todo el poder encierra &
        en lo estrecho de un Pesebre? \&

        \Stanza{4}
        El Portal no era nada, &
        y al ver que hoy tiene &
        en su albergue a Dios Niño, &
        se desvanece. \&

        \Stanza{4}
        Que será que a los Pastores &
        un Paraninfo recuerde, &
        y dejando sus ovejas &
        hoy más ganados se encuentren? \&

        \Stanza{4}
        Es Pastor, y así el Niño &
        quiere premiarlos, &
        porque cuiden gustosos &
        de sus Rebaños.	\&

        \Stanza{4}
        Que será que una Doncella &
        tenga un Infante en su albergue, &
        siendo a un tiempo Madre y Virgen, &
        sin que uno, ni otro se niegue?	\&

        \Stanza{4}
        Es muy justa la duda, &	
        pero yo entiendo, &
        que salir de ella puedes &
        dentro de un Credo. \&

        \Stanza{4}
        Que será que hasta dos Brutos &	
        se precien tan de corteses, &
        que entre los dos la Verdad &
        hoy se mira solamente? \&

        \Stanza{4}
        Tienen a Dios delante, &	
        y así es muy justo, &
        que la modestia se halle &
        hasta en los Brutos. \&

        \Stanza{4}
        Que será que los arroyos &
        hoy sus cristales detienen &
        y se muestran más gustosos &
        cuando están menos corrientes? \&

        \Stanza{4}
        Como el Río de Gracia &	
        sale hoy de Madre, &
        a su vista las aguas &
        vierten cristales. \&
    \end{original}

    \begin{translation}
        \StanzaSection{4}
        What heavenly music &
        is that which alters the air today? &
        What sovereign harmony &
        is that which elevates hearing? \&

        \Stanza{4}
        What light is this that transforms &
        the dense night into day? &
        What clear splendor does the sky &
        this night give to the earth? \&

        \Stanza{2}
        What can be seen of this riddle &
        encloses a great new thing. \&

        \Stanza{4}
        Glory---let the voices repeat it. &
        Peace---their lights give cadences. &
        All the earth has become heaven &
        and all heaven appears on earth. \&

        \Stanza{4}
        What should it mean that in our doubt &
        there is no room to know what it encloses, &
        since the heaven is all voices, &
        since Glorias are in all the earth.
        \SectionBreak

        \Stanza{6}
        \emph{Angel:} The cause, shepherds, is &
        that from a maid &
        the Word takes form in flesh, &
        to pay the debt &
        that from the first Father &
        He took upon his account: \&

        \Stanza{6}
        and the sky, seeing with envy &
        that today the earth &
        knows God as a human, &
        with loud voices celebrates &
        the saying that the Orb &
        holds in its sphere.
        \SectionBreak

        \Stanza{1}
        Let him be welcomed.
        \SectionBreak

        \Stanza{3}
        For if he comes to give us &
        glory instead of grief, &
        let him be welcomed. \&

        \Stanza{4}
        And of our doubts &
        go on with the theme, &
        for in substance &
        I will tell you what the doubts enclose. \&

        \Stanza{4}
        They are admirations, &
        seeing that on the earth &
        luminaries are sown &
        and stars are gathered. \&

        \Stanza{4}
        Go on with the questions, &
        and in solemen cadences &
        with resonant voices &
        I will decipher the theme.
        \SectionBreak

        \Stanza{1}
        Let him be welcomed.
        \SectionBreak

        \StanzaSection{4}
        What shall it mean that in grave voices &
        all the heavenly court &
        with \quoted{glory and peace} banquets us &
        in alternated motets? \&

        \Stanza{4}
        It means that the sky today joyously &
        (from such glories) &
        breaks open with so much finery &
        of accents in the air. \&

        \Stanza{4}
        What shall it mean that at midnight &
        through the portals of the East, &
        the fair Dawn not breaking, &
        the Sun shows himself to us? \&

        \Stanza{4}
        It means that the Sun, when he is born &
        from such an Aurora, &
        gilds the error of Man &
        just as \critnote{he adores her}
        {Play on \foreign{adora} (adores) and \foreign{dora} (gilds), in contrast
        to \foreign{yerro} (error) and \foreign{hierro} (iron).} \&

        \Stanza{4}
        What shall it mean that in a little stall &
        such a presence is hidden &
        that the stable encloses all Power &
        in the narrow bed of a manger? \&

        \Stanza{4}
        The stall was nothing, &
        and on seeing that this day it holds &
        in its lodging God as a Baby, &
        it vanishes. \&

        \Stanza{4}
        What shall it mean that a heavenly sentinel &
        has regard for the shepherds,  &
        and, leaving their sheep,  &
        this day more flocks are gathered? \&

        \Stanza{4}
        He is a shepherd, and thus the baby &
        wants to reward them, &
        so that they should properly care  & 
        for his flocks. \&

        \Stanza{4}
        What shall it mean that a maiden  &
        should have an infant in her lodging, &
        being at once mother and virgin,  &
        without either one negating the other? \&

        \Stanza{4}
        The doubt is very reasonable,  &
        but I understand  &
        that you can come out from her & 
        before I can say the Creed. \&

        \Stanza{4}
        What shall it mean that even two beasts  &
        conduct themselves with such courtesies  &
        that between the two only the Truth  &
        this day is seen? \&

        \Stanza{4}
        They have God before them,  &
        and thus it is very right  &
        that modesty should be found  &
        even among the beasts. \&

        \Stanza{4}
        What shall it mean that the streams  &
        this day hold back their crystal drops  &
        and show themselves to be the most tasteful  &
        when they are the least current? \&

        \Stanza{4}
        As the river of grace  &
        comes forth today from his mother,  &
        on seeing him the waters  &
        turn into crystals. \&
    \end{translation}
\end{poemtranslation}
\endinput
