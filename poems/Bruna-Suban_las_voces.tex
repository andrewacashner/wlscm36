% Suban las voces, setting I by Pablo Bruna
\begin{poemtitleblock}
    \poemhead{\wtitle{Suban las voces al cielo} (Daroca, ca. 1650)}
    \poemsource{%
        Anonymous, from musical setting by Pablo Bruna
        (\signature{E-Bbc}{M759/44}; Girona Cathedral, unnumbered MS, ed.
        Pedro Calahorra Martínez)%
    }
\end{poemtitleblock}

\begin{poemtranslation}
    \begin{original}
        \StanzaSection{4}[\add{Estribillo}]
        Suban las voces al cielo &
        y digan que en esta mesa &
        fénix se abrasa un alma	 &
        de amores llena. \&

        \Stanza{7}
        Y mudando el aire &
        en veloces corcheas, &
        vuelen, vuelen juntas &
        en síncopas que elevan &
        y en bemoles blandos, &
        trinados que suspendan, &
        digan en paso todas: \&

        \Stanza{2}
        Ay, que se abrasa un alma, &
        ay, que se quema.
        \SectionBreak

        \StanzaSection{4}[Coplas \add{Barcelona and Girona}]
        B1/G1. Fénix hermoso eres, alma, &
        que entre cenizas renaces;	 &
        si en Dios hallas nueva vida,	 &
        \emph{arde}. 			 
        \SectionBreak

        \StanzaSection{4}[\add{Barcelona only}]
        B2. De tu mismo fin procedes	 &
        para siempre eternizarte;	 &
        si está tu ser en no ser,	 &
        \emph{arde}. 			 \&

        \Stanza{4}
        B3. En el peligro más cierto	 &
        hallas lisonjas los males;	 &
        si en el fuego no peligras,	 &
        \emph{arde}. 			 \&

        \Stanza{4}
        B4. A la vida lisonjeas		 &
        cuando llegas a abrasarte;	 &
        si con morir te eternizas,	 &
        \emph{arde}. 			 \&

        \Stanza{4}
        B5. Cuando el fuego te consume,	 &
        a tumba es cuna en que naces;	 &
        si del polvo resucitas,		 &
        \emph{arde}. 			 \&

        \Stanza{4}
        B6. A mejor Arabia fénix	 &
        hoy arrepentida partes;		 &
        si la vida está en la muerte,	 &
        \emph{arde}.
        \SectionBreak

        \StanzaSection{5}[\add{Girona only}]
        G2. Lo vivido ya no es vida	 &
        pues muriendo se acabó;		 &
        y lo que queda quién sabe	 &
        si dejará de ser hoy,		 &
        \emph{arde}. 			 \&

        \Stanza{5}
        G3. Alma en el camino estás,	 &
        mira aquella luz, por Dios,	 &
        que es lástima y aun desdicha	 &
        perderse con tanto sol:		 &
        \emph{arde}. 			 \&
    \end{original}

    \begin{translation}
        \Stanza{4}
        Let the voices ascend to Heaven, &
        and let them say that on this table &
        a phoenix is consumed, a soul &
        full of love. \&

        \Stanza{7}
        And transforming the air &
        into rapid quavers, &
        let them fly, fly together &
        in syncopations that they raise &
        and in mild flats, &
        trills that they suspend, &
        let them all say in time together: \&

        \Stanza{2}
        Ah, a soul consumed in flames, &
        ah, a soul that burns.
        \SectionBreak

        \StanzaSection{4}
        B1/G1. O soul, you are a handsome phoenix, &
        who are reborn among the ashes;	        &
        if you should find new life in God,	&
        \emph{burn}.	                        \&

        \StanzaSection{4}
        B2. From your very end, you go forth	&
        to become eternal forever;	        &
        if your being is in not being,	        &
        \emph{burn}.	                        \&

        \Stanza{4}
        B3. In the most certain danger	        &
        you find evils to be flatteries;        &
        if in the fire you are not imperiled,	&
        \emph{burn}.	                        \&

        \Stanza{4}
        B4. You flatter life	                &
        when you come to be consumed in flames;	&
        if in dying you become eternal,	        &
        \emph{burn}.	                        \&

        \Stanza{4}
        B5. When the fire consumes you,	        &
        the tomb is the cradle in which you are born; &
        if from the dust you resurrect,	        &
        \emph{burn}.	                        \&

        \Stanza{4}
        B6. To greater Arabia, phoenix,	        &
        today you depart repentant/suddenly;	&
        if life is in death,	                &
        \emph{burn}.	                        \&

        \StanzaSection{5}
        G2. The life lived is no longer life	&
        for in dying it is accomplished;	&
        and of what remains, who knows	        &
        if it will cease to be today?	        &
        \emph{burn}.	                        \&

        \Stanza{5}
        G3. Soul, you are on the road,	        &
        look at that light, for God's sake,	&
        for it is a shame and indeed a disgrace	&
        to be lost when there is so much sun:	&
        \emph{burn}.	                        \&

    \end{translation}
\end{poemtranslation}
\endinput

