% Carrion, Qué destemplada armonía, E-SE: 20/5
% 2020/05/07
\begin{poemtitleblock}
    \poemhead{\wtitle{Qué destemplada armonía} (Segovia, ca. 1690)}
    \poemsource{%
        Anonymous, from musical setting by Jerónimo de Carrión
        (\signature{E-SE}{20/5})%
    }
\end{poemtitleblock}

\begin{poemtranslation}
    \begin{original}
        \StanzaSection{4}[Estribillo]
        Qué destemplada armonía &
        de confusas voces varias &
        de lo profundo del valle &
        la sagrada esfera escala? \&

        \Stanza{4}
        Sin duda gime oprimida &
        la naturaleza humana &
        al peso de infiel cadena &
        que honrosamente arrastra. \&

        \Stanza{4}
        Pues con ayes tristes, &
        pues con tiernas ansias &
        el cielo penetra &
        y a su piedad clama.
        \SectionBreak

        \Stanza{4}
        Y elevando los dulces acentos &
        cuando le asegura feliz esperanza, &
        al eterno le pide rendida &
        el cumplimiento de su alta palabra. \&

        \Stanza{5}
        Y así dice, y así exclama, &
        ¿hasta cuándo, Señor poderoso, &
        dueño de las almas, &
        han de estar de tus misericordias &
        las puertas cerradas?
        \SectionBreak

        \Stanza{4}
        Pero ya del Olimpo de luces &
        la risa del alba &
        nos anuncia en el sol &
        que previene la paz deseada.
        \SectionBreak

        \Stanza{1}
        Venga, llueva, llegue, nazca.
        \SectionBreak

        \Stanza{4}
        Venga pues en buen hora &
        donde le aguardan &
        a borrar feas culpas, &
        a lavar manchas. \&

        \Stanza{4}
        Llueva la prodigiosa &
        nube sagrada, &
        el suave rocío &
        que es todo gracia. \&
       
        \Stanza{4}
        Llegue aquel luminoso &
        sol que en sus alas &
        hoy para los mortales &
        la salud traiga. \&

        \Stanza{4}
        Nazca al mundo el dorado &
        granito en pajas, &
        porque sea del hombre &
        dulce vianda.
        \SectionBreak
        
        \Stanza{1}
        Venga, llueva, llegue, nazca.
        \SectionBreak


        \StanzaSection{4}[Coplas]
        1. Venga aquel capitan fuerte &
        a cuya diestra bizarra &
        debe Israel tantos triunfos &
        como prometen sus palmas. \&

        \Stanza{4}
        Ánime huestes, &
        aliste escuadras, &
        logre coronas, &
        venza batallas. \&
       
        \Stanza{1}
        Venga, llueva, llegue, nazca.
        \SectionBreak

        \Stanza{4}
        2. Llueva aquel blando rocío, &
        nube hermosa siempre intacta, &
        que el erial campo fecunde & % XXX erial?
        que agostó la culpa infausta. \& %% agostó?

        \Stanza{4}
        Desate en perlas &
        sus abundancias &
        con que corone  &
        las esperanzas. \&
       
        \Stanza{1}
        Venga, llueva, llegue, nazca.
        \SectionBreak

        \Stanza{4}
        3. Llegue aquella luz propicia &
        de profetas deseada, &
        a cuyo influjo benigno &
        se vivifiquen las almas. \&

        \Stanza{4}
        Aliente rayos &
        de luz y llama, &
        que el mundo ilustren &
        cuando le abrasan. \&
       
       \Stanza{1}
        Venga, llueva, llegue, nazca. 
        \SectionBreak

        \Stanza{4}
        4. Nazca y en su infancia tierna &
        sus manos no aten las fajas, &
        porque sus misericordias &
        a dos manos las reparta. \&

        \Stanza{4}
        Del alto seno &
        de eterna estancia &
        desciende al valle &
        que ya le aguarda. \&

        \Stanza{1}
        Venga, llueva, llegue, nazca. \&
    \end{original}

    \begin{translation}
        \StanzaSection{4}
        What untempered harmony &
        from so many confused voices &
        rises from the depth of the valley &
        and scales the sacred sphere? \&

        \Stanza{4}
        No doubt, human nature &
        groans, oppressed &
        under the weight of the traitorous chain &
        that honorably restrains it. \&

        \Stanza{4}
        But now with sad sighs, &
        now with tender plaints, &
        she breaks through heaven & % XXX she = human nature
        and makes a claim on his faithfulness.
        \SectionBreak

        \Stanza{4}
        And lifting up sweet accents &
        when cheerful hope reassures her, &
        on bended knee she petitions the Eternal &
        for the fulfillment of his lofty word. \&

        \Stanza{5}
        Y thus she speaks, and thus she exclaims, &
        how long, O mighty Lord, &
        master of souls, &
        will the gates of your mercies &
        remain closed? 
        \SectionBreak

        \Stanza{4}
        But already from the Olympus of lights &
        the laughter of the dawn &
        announces to us in the sun &
        that comes before the longed-for peace.
        \SectionBreak

        \Stanza{1}
        Come, rain, arrive, be born.
        \SectionBreak

        \Stanza{4}
        Let him come, then, at the right time &
        when they are waiting for him &
        to wipe away hideous faults, &
        to wash away every stain. \&

        \Stanza{4}
        Let it rain, the ample &
        holy cloud, &
        the mild dew &
        that is full grace. \&

        \Stanza{4}
        Let him arrive, that luminous &
        sun who in his wings &
        today brings healing &
        to mortals. \&

        \Stanza{4}
        Let him be born to the world, &
        the golden seed in the straw, &
        so that he may be for Man &
        sweet nourishment. % XXX vianda?
        \SectionBreak

        \Stanza{1}
        Come, rain, arrive, be born.
        \SectionBreak

        \StanzaSection{4}
        1. Let him come, that strong captain &
        to whose awesome fighting skill &
        Israel owes so many triumphs, &
        as his palms foretold. \&

        \Stanza{4}
        Rise up, hosts, &
        move out, squadrons, &
        earn laurels, &
        win battles. \&

        \Stanza{1}
        Come, rain, arrive, be born.
        \SectionBreak

        \Stanza{4}
        2. Let her rain, that gentle dew &
        lovely cloud, ever whole, &
        who waters the fertile field, &
        who overcomes the guilt she never knew. \& % XXX ???

        \Stanza{4}
        Untie in pearls &
        her abundant riches &
        with which she crowns &
        every hope. \&

        \Stanza{1}
        Come, rain, arrive, be born.
        \SectionBreak

        \Stanza{4}
        3. Let it come, that clear light & %XXX propicia
        desired by prophets, &
        by whose benign influence &
        souls are brought to life. \&

        \Stanza{4}
        Let the rays shine forth &
        of light and flame, &
        to illuminate the world &
        when they set it afire. \& % XXX

        \Stanza{1}
        Come, rain, arrive, be born.
        \SectionBreak
        
        \Stanza{4}
        4. Let him be born, and in his tender infancy &
        his hands do not tie up the bundles of straw, &
        because his mercies &
        with two hands divide them. \&

        \Stanza{4}
        From the high mound & %XXX seno
        of eternal dwelling &
        he descends to the valley &
        that already awaits him. \&

        \Stanza{1}
        Come, rain, arrive, be born.
        \SectionBreak
    \end{translation}
\end{poemtranslation}
\endinput
