% poem.tex
% AFUERA, AFUERA, PASTORES (JACARA)
% POEM AS SET BY JUAN GUTIERREZ DE PADILLA
% MEX-Pc: Leg. 1/3 (1652)
% EDITED BY ANDREW A. CASHNER

% 2017/05/19 transcription
% 2020/05/07 translation

\begin{poemtitleblock}
    \poemhead{\worktitle{Afuera, afuera pastores (Jácara)} (Puebla, 1652)}
    \poemsource{%
        Adapted from Feliz Persio Bertiso (Seville, 1634,
        \signature{MEX-Plf}{80070-42010404}), 
        from musical setting by Juan Gutiérrez de Padilla
        (\signature{MEX-Pc}{Leg. 1/3})%
    }
\end{poemtitleblock}

\begin{poemtranslation}
    \begin{original}
        \StanzaSection{4}[Jácara \add{Coplas}]
        1. Afuera, afuera, pastores, &
        zagalejo aparta, aparta, &
        que el sol de la valentía &
        viene a ver parida el alba. \& 
        % XXX a ver = haber?

        \Stanza{4}
        2. Apártense del pesebre, &
        veremos el talle y gala, &
        destas pajas venturosas, &
        cuerpo de Dios \critnote{con las pajas}
        {In several idioms \term{pajas} can also mean boasts,
        cuing the listener to the rogues' bragging talk to follow}. \&

        \Stanza{4}
        3. Afuera todo \critnote{zamarro}
        {MS: \term{samarro}, Phonetic spelling of Andalusian/New Spanish pronunciation} &  
        que viene la gurullada &  
        de los bravos macarenos & 
        a ocupar esta cabaña. \&

        \Stanza{4}
        4. Rey mío recién nacido, &
        de quien tiembla y \critnote{sea milana}
        {Meaning unknown} & 
        la flor de la valentía, &
        sin desgarros ni \critnote{el haracas}
        {Meaning unknown}. \&

        \Stanza{4}
        5. Por Dios verdadero os tengo, &
        y lo juraré en un ara, &
        \critnote{no sé}[no sé cómo se os atreve/ la señor doña escarcha]
        {Precise meaning unclear}
        cómo se os atreve, &
        la señora doña escarcha. \&

        \Stanza{4}
        6. Rey sois de cielos y tierra, & 
        por parte de \critnote{taita y nana}
        {Daddy and mommy, referring to David's royal line} & 
        y no nacen desta suerte &
        los reyes allá en España. \&

        \Stanza{4}
        7. Lindo palacio y tapices &
        no os dieran esta posada &
        si nacierais entre herejes &
        allá en Holanda o \critnote{Jelanda}
        {Probably a joke place name}. \& 

        \Stanza{4}
        8. Ésta es afrenta, Rey mío, &
        de los bravos de la hampa, &
        quítense las capas todos, &
        ropa afuera, camaradas. \&

        \Stanza{4}
        9. Hagamos de ellas al niño, &
        un \critnote{pavellón}
        {A cloth hung up to shelter a bed, like a tarp when camping}
        y una cama, & 
        pues vemos que estos judíos &
        ya como quien son lo tratan. \&

        \Stanza{4}
        10. Y pues con sus capas ellos &
        lo han de honrar cuando se vaya & 
        nosotros hoy cuando viene, &
        honrémosle con las capas. \&

        \Stanza{4}
        11. Sirva de colchón y colcha, &
        mi \critnote{capote de campaña}
        {Like a poncho, a heavy blanket that doubled as cloak and bedding for
        soldiers; tailor's patterns for these may be found in Martín de
        Anduxar, \wtitle{Geometria y trazas pertenecientes al oficio de
        sastres} (Madrid, 1640), 20}, &
        que capote de serranos &
        allá para las bacadas. \&

        \Stanza{4}
        12. Aqueste \critnote{colento}
        {Meaning unknown} 
        de ante, &
        sirva de alfombra \critnote{bizarro}
        {MS: \term{bissaro}, phonetic spelling of a common key word in
        \term{jácaras}}, & 
        que es rendiros a los pies, &
        la defensa de las armas. \&

        \Stanza{4}
        13. \critnote{Y comida sella el hielo}
        {MS: \term{Y comida se ya el yelo}, meaning uncertain}, &
        y esta nieve silla helada &
        deje sus atrevimientos &
        o háralo a cuchilladas. \&

        \Stanza{4}
        14. Paréceme que os dormís & 
        hola compañeros, vaya &
        un tono que al niño sirva &
        de dormitorio y hamaca. \&

        \Stanza{4}
        15. Rómpanse los instrumentos &
        y démonos de las astas, & 
        que ésta es la jacarandina, &
        y ésta es la jacarandana.
        \SectionBreak

        \StanzaSection{10}[Estribillo]
        Pues de Dios las bravezas &
        se han humanado, &
        ríndansele hampones, &
        sírvanle bravos, &
        pastores, hola, &
        callen esos \critnote{adufes}
        {Square skin drums still played by Iberian folk musicians} &
        y esas \critnote{zampoñas}
        {MS: \term{sampoñas}, phonetic spelling} &
        oíganse, digo. &
        Miren que está durmiendo, &
        cuerpo de Cristo. \&
    \end{original}

    \begin{translation}
        \StanzaSection{4}
        1. \critnote{Make way, make way, shepherds}
        {First of several indications of antagonism between the shepherds and
        the \term{jaques} or rogues (outlaws, ruffians) who come after them to
        the manger}, &
        herder-boy, clear out, clear out: &
        the \critnote{sun}[sun\dots{} dawn]
        {The dawn is the Virgin Mary and the sun that rises from the dawn is
        Christ} of valiance is coming, &
        to see that the the dawn has given birth. \&

        \Stanza{4}
        2. Stand back from the manger, &
        \critnote{let's size up the shape and dress}
        {They want to take the measure (\term{talle}, as in tailoring) and
        check out the decorations (\term{gala}) of the Christ-child's
        accommodations; seeing their poverty the ruffians are going to supply
        what they can from their capes and other gear, below} &
        made from this lucky straw, &
        the body of God among the straw. \& 

        \Stanza{4}
        3. Make way, all you \critnote{tough guys}
        {\term{Zamarro}, a crude, uncouth, rough-living person}: & 
        for here comes the \critnote{squad}
        {One of many plays on military jargon} &
        of \critnote{macho}
        {\term{Macareno}, affecting bravery, boastful} 
        braves &
        to occupy these barracks. \&

        \Stanza{4}
        4. My newborn king, &
        who is making you tremble? & 
        when you are the flower of valiance, &
        no need to boast. \& 

        \Stanza{4}
        5. I regard you as the true God &
        and I will swear to it on an altar, &
        I don't know how \add{the cold} defies you--- &
        it chills the noble lady. \&

        \Stanza{4}
        6. You are the king of the heavens and earth, & 
        on account of your daddy and mommy, &
        and the kings over there in Spain &
        are not born in this manner. \&

        \Stanza{4}
        7. A lovely palace and tapestries--- &
        they would not have given you \emph{this} dwelling &
        \critnote{if you had been born among heretics}
        {In other words, even heretical Christians (Dutch Protestants) would
        have known to give Christ a more suitable royal dwelling than the Jews
        did (see copla 9)}&
        there in Holland or Hayland. \& 

        \Stanza{4}
        8. This is an affront, my king, &
        from this fierce group of criminals. &
        Take off your cloaks, everyone--- &
        capes off, comrades. \&

        \Stanza{4}
        9. Let's make from these clothes &
        a tent and a bed for the child, & 
        since we see that \critnote{these Jews}
        {Anti-Semitic innuendo directed perhaps at the innkeeper who was
        responsible for the poor lodgings, and presumably not at the Mary and
        Joseph, who of course were also Jewish} &
        treat him as they do. \&

        \Stanza{4}
        10. And since they are going to &
        \critnote{honor him with their capes when he leaves}
        {Referring to Christ's Palm Sunday triumphal entry to Jerusalem on a
        bed of cloaks and branches}, &
        let us today, when he is coming, &
        honor him with the capes. \&

        \Stanza{4}
        11. Let this serve as a bedroll and blanket, & 
        the soldier's cloak from my uniform, & 
        and use the mountain-folk's cloaks &
        over there for the cattle. \&

        \Stanza{4}
        12. Let this skin garment & 
        serve as a ruffian's blanket; &
        with this we lay down at your feet &
        the defense of arms. \&

        \Stanza{4}
        13. And let the ice preserve the food &
        and this snow a frozen chair; &
        \critnote{leave off your insults and boasts}
        {Having asked his fellow rogues to offer the child their garments now
        he tries to reform their bad manners} &
        or do it at the point of a sword. \&

        \Stanza{4}
        14. It looks to me like you are sleeping; &
        hey fellows, let's have &
        a song that can serve for the child &
        as a room and a hammock. \& 

        \Stanza{4}
        15. \critnote{Dash those instruments}
        {Speaking to the shepherds, who in villancicos as in exegetical
        literature were always singing and playing on their way to the manger}, &
        give us your staffs, & 
        for this here is the rogue's song, &
        and this is way rogues talk.
        \SectionBreak

        \StanzaSection{10}
        Since from God all bravery &
        has been made human, &
        bow before him, all you outlaws, &
        let the rogues serve him; &
        shepherds, hey, &
        shut up those square drums & 
        and those panpipes, &
        listen, I tell you. &
        Look, for he is sleeping, &
        the body of Christ. \&
    \end{translation}
\end{poemtranslation}
\endinput

