This edition comprises the second volume of critical editions of villancico
poetry and music that were the subject of my monograph, \wtitle{Hearing Faith:
Music as Theology in the Spanish Empire} (Leiden: Brill, 2020).
Readers are referred to the first volume for an introduction to the
villancico genre, a full bibliography of relevant research, and
recommendations for performance practice.%
    \Autocite{Cashner:WLSCM32}
This volume applies the same editorial policies outlined there.
The following comments are abstracted from the monograph.

The devotional music collected in this edition is all in some way
\quoted{music about music}---the poetic text and the musical setting refer
listeners to other kinds of music beyond the sounds they are hearing in the
moment of performance.
The pieces invoke the music of birds, human voices and instruments, the
celestial spheres, the angelic chorus, and the supreme theological
\quoted{music} of the Triune God.
\quoted{Metamusical} pieces like these invited hearers to listen beyond their
natural hearing, to strain to hear a higher theological kind of music, within
a Neoplatonic theological system shaped by Augustine and Boethius.

One way to do this was to construct a poem out of musical terms, and then
embody those references in the musical setting.
Juan Gutiérrez de Padilla's \wtitle{Miraba el sol el águila bella} is a
virtuosic demonstration of this craft.
The whole text is built out of Guidonian solmization syllables (\term{ut, re,
mi, fa, sol, la}).
In the surviving Tenor part, the chapelmaster of Puebla Cathedral in New Spain
has set every syllable to a pitch that could be sung to that syllable, so that
singing the text and solmizing the tune are nearly the same thing.
One might expect such a text to be nonsense, but remarkably, the poem actually
constructs an extended \emph{triple} conceit (in the tradition of Spanish poetic
\term{conceptismo}) about the Virgin Mary as Immaculate, the eagle looking at
directly the sun, and music.
In traditional lore, the eagle was believed to have the capacity to stare
directly at the sun without being harmed; its enemy was the kite
(\foreign{milano}).
This fragment is preserved in an anonymous handwritten collection in the
archive of Puebla Cathedral, a personal anthology written by an untrained hand
including parts from masses of Palestrina and a variety of \quoted{villancicos
by various authors}.
It is not hard to imagine why the scribe wanted to make this copy of Gutiérrez
de Padilla's music (the rest of which is apparently lost), and for some of the
same reasons teachers today might use the piece today to demonstrate the
technique of Guidonian solmization, which was clearly still a living practice
in mid-seventeenth century Spanish America.

Poetic texts like this were also favorite choices for musical setting by
composers in the Spanish Empire because they enabled them to show off their
musical and theological dexterity in matching the discourse about music in the
poem with actual music.
Especially by setting the same or similar texts as those previously set
by teachers, paragons, or rivals, chapelmasters established their position
within a lineage of metamusical composition.
The clearest example of this pattern of imitation in this edition is the pair
of settings of a text beginning \wtitle{Suban las voces al cielo} by Pablo
Bruna and Miguel Ambiela.
These linked pieces were first brought to light by Pedro Calahorra.%
    \citXXX[Calahorra]
This edition draws on a previously undiscovered source of the Bruna villancico
and corrects serious errors in the Ambiela.
Bruna (\XXX[dates]) was an acclaimed organist, blind from birth, in the
Spanish (Aragonese) town of Daroca near Zaragoza.
Bruna's setting is preserved in two sets of manuscript performing parts, one
in the archive of Girona Cathedral (the basis of Calahorra's edition), and
another in Barcelona's Biblioteca de Catalunya, previously unattributed.%
\begin{Footnote}
    I regret that I was not able to view the Girona manuscript, so I had to
    rely on Calahorra's edition for that source.
\end{Footnote}
A few years after Bruna's death, a young Ambiela was sent to study in Daroca
and later served briefly as chapelmaster there, before ascending to some of
the most prestigious posts on the peninsula.
Whether he encountered Bruna's \wtitle{Suban las voces} in the archive or as a
performer, he composed an homage to it by setting an altered version of the
text.
The manuscript performing parts bear a copying date of 1689 in the hand of an
idle choirboy, apparently a member of the choir in Lérida while Ambiela was
chapelmaster there, who wrote some derisive slurs about Daroca on the cover
and filled the rest of the space doodling his name (Torrente).
Ambiela quotes motives and borrows its formal plan loosely from Bruna, but the
homage is actually most apparent in the places where the new pieces differs
from the model.
In multiple instances Ambiela tries to outdo his predecessor---more voices,
more extravagant gestures, overlapping lines of text.
The most telling passage is the setting of \foreign{bemoles blandos} (mild
flats).
Bruna does what any Spanish composer of his time would have done for a
metamusical text like this, and writes a passage filled with flats.
But Ambiela must go beyond: he counters by setting the same words with music
all in \emph{sharps}. 
Without the model of Bruna, the younger composer surely would have done the
obvious thing; his choice to represent the meaning by depicting its opposite
shows the strain of imitation, even the burden of needing to find ever-new
ways to use music to represent itself.

A similar progression between composers in a lineage may be seen in the pair
of villancicos by Miguel de Irízar and Jerónimo de Carrión \XXX[dates],
successive chapelmasters of Segovia Cathedral (another pair of related works
is edited in volume I).
Both pieces are \term{villancicos de calenda}, the opening piece in the set of
villancicos performed for Christmas, and both composers show off the full
range of their polychoral ensemble, which clearly featured several skilled
soloists.
The scores of Irízar's drafts survive---among very few extant villancicos in
score format---in makeshift notebooks made from piles of Irízar's received
letters.
The thrify composer squeezed in up to \XXX[eleven] staves of music on the
empty sides and margins of the letters.
The letters largely concern the exchange of villancico poems and music among a
network of affiliated Spanish musicians, including Pedro de Ardanaz in Toledo,
Diego de Cáseda in Zaragoza, \XXX[Miguel] Xúarez in Seville, and Miguel Mateo
de Dallo y Lana, who emigrated from Seville to Puebla and likely brought music
from this same network to the New World.
The letters make it possible to determine the sources of every villancico for
Irízar's 1678 Christmas cycle.
He received the text for \wtitle{Qué música celestial} in the mail from his
friend Ardanaz, a fellow pupil of Tomás de Miciezes \XXX[Sr?], after Ardanaz
had set it to music the previous Christmas at Toledo Cathedral.
Irízar combined several of Ardanaz's selections with other works recently
performed in Madrid and elsewhere and then crafted his own music to meet local
needs, working under a tight deadline---the latest letter in the packet is
dated September 8, 1678, allowing Irízar at most a week per piece, assuming
only limited rehearsal time.

Carrión's piece has the larger and more sectionalized dimensions of a late
seventeenth-century villancico, which becomes similar to a \term{cantada},
with extended solo passages.
The estribillo is now so long that only a fragment of the end is repeated
after the coplas.
(Ambiela had already experimented with novel ways of repeating only a portion
of the estribillo.)
Theologically Irízar's piece focuses on evoking wonder at the mysterious music
of heaven come down to earth---referring both to the angel chorus heard by the
Bethlehem shepherds and to the harmony of divine and human in the Incarnation
of Jesus.
Carrión's piece moves the other direction: the untuneful, distempered harmony
of sinful humanity rises to heaven and through faith appeals to God's mercy,
who in response sends Christ to be born, like rain pouring down after a
drought.

The remaining villancico in this edition, Gutiérrez de Padilla's \term{jácara}
of 1652, is less explicitly about music apart from the final copla, but it
represents a popular and puzzling subgenre of villancico that purports to
embody a specific kind of dance-song.
\term{Jácaras} originated as a type of poetic, musical, and dance performance
in Spanish theaters and on the street, in which singers recounted the exploits
of \term{jaques}---outlaws, rogues, ruffians---using the rough language of
criminals and soldiers, accompanied with dances that were condemned by
authorities for being sexually suggestive and obscene.%
    \citXXX[Torrente]
Around the 1630s, poets began adapting the genre \term{a lo divino}, creating
a new \quoted{sacred \term{jácara}} in which the role of the renowned outlaw
was taken by the baby Jesus, his mother, or other holy figures.
Juan Gutiérrez de Padilla included a \term{jácara} in each of his Christmas
cycles for the newly consecrated Puebla Cathedral from 1651--1657
\XXX[check].%
    \citXXX[recording]
He adapted this text from one by Feliz Persio Bertiso in the poetry imprint
commemorating a 1634 Seville performance, the only copy of which survives
today in a binder's collection in Puebla's Biblioteca Lafragua.%
    \footnote{\signature{MEX-Plf: 80070-42010404}.}
Each of his settings starts from the same basic tune and groove that Álvaro
Torrente has reconstructed for the secular \term{jácara}, but the Puebla
chapelmaster uses the resources of his virtuoso polychoral ensemble to
increase the musical complexity each year, culminating in the justly famous
\term{A la jacara, jacarilla} of 1655.
The play with intricate triple-meter polyrhythms matches the verbal ingenuity
of the genre and helps to embody the dexterity and skill ascribed to a 
heroic warrior.

Performers, please take note of the racist language in several of the
coplas, including insults directed at Jews, Dutch Protestants, and possibly a
reference to African slaves, and consider omitting these verses in public
performance.
Gutiérrez de Padilla was himself a slaveholder as well as an Oratorian priest,
and he perpetuates the typical prejudices of men of his social class; we owe
our audiences better than to revive these notions or pretend that they do not
matter.%
    \citXXX[sources]

Happily, the other pieces contain little that could cause offense except
perhaps to those with an anti-religious or anti-Catholic bias. 
These pieces, including (somehow) the \term{jácara}, were created to function
as acts of worship, inciting faithful hearers to increased faith and devotion.
By drawing listener's attention to the link between faith and hearing, they
challenged them to reflect on the process of listening and to use music as a
means to encounter the divine.

This edition is based on archival research in Puebla, Mexico City, Barcelona,
Madrid, and Segovia, in 2012.
This research was supported by a Jacob K. Javits Fellowship from the United
States Department of Education, an ACLS/Mellon Dissertation Completion
Fellowship, and travel grants from the University of Chicago Center for Latin
American Studies, Columbia University's Center for European Studies, and the
American Musicological Society.
The final stage of preparation was supported by a Humanities Center Fellowship
from the University of Rochester.
The edition was prepared using free and open-source software.
The music was typeset in Lilypond using an extensive custom library and the
text was typeset in \LaTeX{} using the editor's own packages,
\term{semantic-markup}, \term{octave}, \term{musicography} and
\term{poemtranslation}.

The complete sources for this edition are in a Git repository hosted on
Bitbucket at \url{http://www.bitbucket.com/andrewacashner/wlscm-v2}; the
Lilypond modules and LaTeX packages are in separate repositories on the same
site.
Please check the editor's website,
\url{http://www.andrewcashner.com/villancicos/}, for updates and corrections,
and please do not hesitate to notify me of any errors.
Performing parts and transposed editions are available upon request.


