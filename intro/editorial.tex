% editorial policies
% 2020/07/06

\section{Abbreviations}

\begin{tabulary}{\textwidth}{lL}
    A. 
    & \term{Alto}, \term{Altus} \\
   
    Ac., Acomp. 
    & \term{Acompañamiento}, \term{basso continuo} \\

    B.
    & \term{Bajo}, \term{Bassus}, Bass \\

    Ch. 
    & Chorus, \term{Coro} \\

    General 
    & \term{Acompañamiento general}, \term{basso continuo} \\

    CN
    & See critical notes \\

    \term{DRAE}
    & \wtitle{Diccionario de la lengua española}, 23rd ed. \\

    Leg. 
    & \term{Legajo}, archival folder \\

    M., mm. 
    & Measure (bar), measures \\

    MS
    & Manuscript \\

    S.
    & Soprano; used for highest voice in part listings to distinguish Tiple
    vs. Tenor (e.g., \quoted{SSAT}) \\

    T.
    & Tenor \\

    Ti.
    & \term{Tiple}: Treble, (boy) soprano \\

    Ti. I-1
    & Chorus I, First Tiple \\
\end{tabulary}
    
\subsection{Pitch and Octave Designations}

This edition uses upper-and lowercase pitch names together with prime symbols
to indicate specific pitches.
These designations map onto Helmholtz octave numbers as follows:
\begin{center}
\octavetable
\end{center}

\subsection{Archival Sigla}

\begin{tabulary}{\textwidth}{llL}
    \textsc{Siglum} & \textsc{Country} & \textsc{Archive} \\

    \emph{E-Bbc}
    & Spain
    & Barcelona, Biblioteca de Catalunya \\

    \emph{E-Mn}
    & Spain
    & Madrid, Biblioteca Nacional de España \\

    \emph{E-SE}
    & Spain
    & Segovia, Catedral, Archivo Capitular \\

    \emph{MEX-Pc}
    & Mexico
    & Puebla, Catedral, Archivo Capitular \\

    \emph{MEX-Plf}
    & Mexico
    & Puebla, Biblioteca José María Lafragua, Benemérita Universidad Autónoma
    de Puebla \\
\end{tabulary}


\section{Editorial Policies}

\subsection{Sources}
The sources for each poem and its musical setting are listed in the 
critical notes.%
\begin{Footnote}
    The sections on editorial policies and performance suggestions are adapted
    from those in volume I: \autocite[5--11]{Cashner:WLSCM32}.
\end{Footnote}
Select images of the sources are included in their own section below.
The music is preserved in individual manuscript performing parts in looseleaf 
sets or bound partbooks.
For the villancico by Miguel de Irízar, the composer's draft score also 
survives.

The texts and translations are based on the poetic text in the musical settings.
They have been annotated and sometimes corrected in comparision with the 
surviving poetry imprints of the same or related villancico poems.
The poems are generally anonymous but are often adapted from existing poems or 
poetic types.

The manuscript parts were practical tools for performers.
They all bear evidence of frequent use over a long period: they are soiled 
along the creases in the paper where performers held them up, and they include 
the names of multiple performers, corrections in different hands, and added 
accidentals and barlines.
Aspects of notation that seem ambiguous to a modern scholar were not, 
apparently, impediments to effective performance from the originals.
The goal of this edition, in keeping with the nature of its sources, is to 
enable the practical performance and study of these villancicos through a clear
and consistent notation.


\subsection{Orthography}

Spelling and punctuation have been modernized and standardized.
Though in doing this some information about historic local pronunciation is
lost, a standard orthography allows performers to present the works in a way
that will be most intelligible to their audiences.%
\begin{Footnote}
    The phonetic orthography in the performing parts does suggest that
    \mentioned{ci} and \mentioned{ce} were pronounced like \mentioned{si} and
    \mentioned{se} in New Spain and Catalonia, rather than with the TH sound in
    modern peninsular Spanish (as in \mentioned{thick}).
\end{Footnote}

\subsection{Translation}

The villancico poems in this edition are complex examples of the Spanish
literary technique of \term{conceptismo}, in which the poem is governed by a
central conceit that links two (or more) ideas together in an extended
metaphor.%
    \Autocites
    [13--14]{Cashner:HearingFaith}
    {Gaylord:Poetry}
In these poems, music forms one side of the conceit, and a theological concept
like Christ's Incarnation or Passion forms the other side, though this is an
oversimplification.
The wording of the Spanish is deliberately ambiguous so that one can read the
poems concentrating on either or both sides of the metaphor.
This means that it nearly impossible to translate the poetry into English and
preserve the delicate balance of double and sometimes triple meanings.
For Spanish words with two meanings, English equivalents with a similar range of
meaning were chosen; but in other cases multiple alternatives had to be
provided.
The translations are as literal as possible while still conveying at least one
level of the original sense.
In some cases, the meaning of a cryptic phrase only becomes clear when read in
the context of contemporary theological and devotional literature.
Perplexed readers are urged to consult the detailed exegesis of these poems in
the editor's monograph, as the translations are based on rigorous textual
criticism and historically grounded contextual interpretation.%
    \Autocite{Cashner:HearingFaith}

\subsection{Voice and Instruments}
The original names for voices and instruments have been preserved (though
spelling has been standardized, so \foreign{Baxo} is rendered as
\foreign{Bajo}).
\mentioned{Tiple} refers to a treble singer, usually a boy.
Several terms are used for continuo parts, such as \term{Acompañamiento},
\term{General}, or \term{Guión}.
The edition preserves indications of solo and instrumental parts when they
appear in the original.
In several pieces the bass parts contain only brief textual incipits because
they were meant to be played instrumentally, and some of them include explicit
instrument names.
Original figured bass is preserved, but continuo realizations are left to the
discretion and creativity of the performer.
Separate instrumental parts and realized keyboard parts are available on request
from the editor.

\subsection{Editorial Text}
Italic text indicates editorial underlay, usually where there are signs in the
sources that specify that the preceding text should be repeated.
Other textual additions by the editor, such as standardized section headings, 
are enclosed in square brackets.

\subsection{Pitch Level}
All pieces are transcribed at their original notated pitch level.
The preparatory staves at the beginning of each piece show the original clefs, 
signatures, and the first note, using symbols close to those used in the
manuscripts.  I traced the CZ symbol in Inkscape from the one used in the
performing parts of Miguel de Irízar from Segovia Cathedral.

More research is needed into Spanish high-clef or \term{chiavette}
conventions, but it seems likely that ensembles did perform many villancicos
at a lower pitch level than notated.
At the same time, the analysis of mode, harmony, \term{musica ficta}, and
musical-rhetorical figures is greatly facilitated by preserving the original
pitch level.
One piece with an exceptionally high written tessitura is also provided in a
transposed edition that may be more practical for modern ensembles.

\subsection{Accidentals}
Accidental placement in the partbooks is contextual and sometimes ambiguous to 
a modern reader.
The original notation has no \na{} symbol, using B\sh{} and E\sh{} instead.
In a few cases, indicated in the critical notes, scribes use a \sh{} sign as a 
cautionary accidental.
One common use was to warn the singer \emph{not} to apply a sharp according 
to \term{musica ficta} conventions.%
  \Autocites
  {Harran:Cautionary1}
  {Harran:Cautionary2}

The edition presents the pitches with their accidental inflections when 
unambiguously specified in at least one source.
According to modern convention, these accidentals are valid until the next 
barline.
Thus repeated accidentals in the source are omitted if the modern convention 
does not require them; and in a few cases accidentals are added where modern 
notation demands.
Editorial suggestions for other accidentals, mostly according to \term{musica 
ficta} conventions, are set above the staff.

\subsection{Repeats}
Some of the sources indicate repeated sections by using barlines with dots (like
modern repeats), or by giving the incipit of the music and text to be repeated;
often there is also a \term{signum congruentiae} at the point of repetition or a
textual note.
In most cases, the estribillo was reprised after the last copla was sung (like
the respond in a Responsory chant).
Some pieces call for a reprise after each copla or after certain groups of
coplas.
In many sources, the repeat of the estribillo is not specified, and it is
possible that it was not always reprised, especially as villancicos became
longer and more complex.%
    \Autocite{Torrente:Estribillo}
This edition uses modern repeat barlines for short repeated sections and
indications of \quoted{D.C. al Fine} or \quoted{D.S. al Fine}, though these
Italian texts are not used in the originals.

\subsection{Rhythm, Meter, Tempo}
The original music was written in mensural notation, with few barlines in the 
performing parts.%
\begin{Footnote}
    Spanish composers like Miguel de Irízar did use barlines when they notated in 
    score format.
    Irízar writes two \term{compases} per bar in both triple and duple meters,
    occasionally squeezing in a third \term{compás} for an odd number
    of groups.
    Cerone advises students who wish to write out a score from parts to write 
    barlines every two \term{compases}; \textcite[745]{Cerone:Melopeo}.
\end{Footnote}
The duple-meter sections of these pieces were written in \meterC{} meter, which
the seventeenth-century Spanish theorists Pedro Cerone and Andrés Lorente refer
to as \term{tiempo menor imperfecto} or \term{compasillo}.%
  \Autocites
  [537]{Cerone:Melopeo}
  [156, 210]{Lorente:Porque}
In this meter, the \term{compás} or \term{tactus} consisted of a semibreve 
divided into two minims.%
  \Autocites
  {GonzalezValle:MusicaTexto}
  {GonzalezValle:CompasCabezon}

The other common meter for seventeenth-century villancicos was notated with
a cursive \meterCZ{} symbol.
Lorente says that this is a shorthand for \meterCThreeTwo{} or \meterCThree{},
where these signs all indicate \term{tiempo menor de proporción menor}, a 
proportion of \meterC{} meter.%
  \autocite[165]{Lorente:Porque}
The \term{compás} consists of one perfect semibreve (\musSemibreveDotted), which
is divided into three minims (\musMinim\ \musMinim\ \musMinim), instead of the
two minims of \meterC{} (\musMinim\ \musMinim).

In the sources, deviations from the normal ternary groups are indicated through 
coloration. 
When noteheads in \meterCThree{} meter are blackened, this often indicates a
shift to \term{sesquialtera} or hemiola.
In \term{sesquialtera} two groups of three minims are exchanged for three 
groups of two minims; and three imperfect semibreves take the place of two
perfect semibreves.

The edition presents the rhythms of the sources according to modern conventions 
of meter and barlines.
The music has been notated in \meterC{} for duple meter and \meterCThree{}
for triple meter.
The original meter signs are shown in preparatory staves or above the staff.
The original note values have not been reduced.
Mensural coloration is indicated with short rectangular brackets above the 
staff.
Ligatures are indicated by long rectangular brackets.
Beaming is unchanged.

In ternary meter, semiminims (quarter notes) are often written as flagged
minims, which are beamed together for melismas in the same way as
\term{corcheas} (\term{fusae}, eighth notes).
I have demarcated these groups with slurs.
Explicit slurs in the original, always indicated melismas and not phrasing,
are preserved, though I have standardized their start and end points.

Regarding tempo, the theoretical $3:2$ proportion of minims between
\meterCThreeTwo{} and \meterC{} does not necessarily imply the same proportion of
tempo.  
In actual practice, a $3:1$ tempo relationship often makes more musical sense,
so that three minims in triple meter together take the same amount of time as
one minim in duple meter.
Thus two \term{compases} of CZ would have about the same duration as one 
\term{compás} of C.

\section{Performance Suggestions}

\subsection{Spanish Pronunciation}
Spanish-speaking ensembles should feel free to pronounce the Spanish according
to their own accent.
Other ensembles are encouraged to work with local native speakers and experts
whenever possible to shape their pronunciation and understanding, so that they
can perform these pieces in a way that Spanish-speaking audience members will
understand and recognize as a part of their own cultural heritage.

\subsection{Instrumentation and Voicing}
These villancicos are scored for an ensemble of voices with instrumental bass 
or continuo groups.
Vocal ensembles varied in size, from one-to-a-part groups to much larger 
polychoral forces.
Most of the pieces also feature prominent solo parts, particularly in the 
\term{coplas}.

The lowest voice parts in these pieces are meant to be performed on instruments. 
They are only provided with short incipits of the text to orient the 
performer, and in several cases instruments like \term{bajón} (dulcian, bass
curtal) or organ are specified.
Though there is need for more research into the specific instrumentation of 
Spanish musical ensembles, it is plausible that the bass line was performed in 
most cases by a continuo group of \term{bajón} doubled by harp, organ, and 
possibly other instruments like the \term{vihuela de mano}.%
    \Autocite
    [On the changing instrumentation in one Spanish institution, see][]
    {Torrente:PhD}
In pieces without figured bass, continuo players---which could include any
polyphonic instruments like keyboard or plucked strings---likely improvised
harmonies to match the other voices.

The upper voices could have been doubled on \term{bajoncillos},
\term{chirimías} (shawms), \term{sacabuches} (sackbuts), and other instruments
according to local resources and suited to the occasion.
There is as yet no clear evidence, though, that church ensembles of 
seventeenth-century Spain or Spanish America included percussion instruments
when performing in the liturgy.%
\begin{Footnote}
    For a critique of exoticizing practices in recent villancico performances,  
    see \autocites{Baker:PerformancePostColonial}{Davies:LocalContent}.
\end{Footnote}

Ensembles should not be deterred by the lack of early instruments or by vocal
ranges outside their resources.
It would be entirely within the spirit of the performing traditions that these
sources represent, for a school or community chorus to substitute modern
instruments for their historic relatives.
At a minimum, it is appropriate to use any keyboard, preferably with bassoon or
cello, for the continuo, and bassoon or cello for the instrumental bass lines.
If more instruments are available, a small organ (or a good digital sample of an 
8$'$ flue-type stop), harp, and classical guitar could be added to the continuo
section.  
Vocal parts could be doubled with bassoons, oboes, trombones, or any other
available instruments.

If possible, it would be appropriate to use soloists or a reduced ensemble for
the first chorus in polychoral pieces, and for the coplas.
In this way a chorus of more modest ability, such as a high school choir, could
be paired with more advanced soloists, such as college students or adult
community members.
If there are more instrumentalists than singers, there should be at least one
singing voice per chorus to present the text.
Instrumental parts and continuo realizations are available from the editor upon
request.

Historic performers made these pieces their own and performed them in a way
that fit their local needs in terms of personnel, instrumentation, acoustic
space, and other factors.
They performed these pieces in a way that was intelligible and meaningful to
them and to their hearers.
Modern performers are continuing in the same spirit when they make practical
adaptations for their circumstances.

\subsection{Ethical Responsibility}
While some amount of adaptation seems appropriate for this repertoire,
performers are urged never to lose sight of the religious, social, and political
contexts of these pieces in their early modern origins.
These villancicos are all devotional pieces, used at some point in liturgical
worship, but they do not fit easily into modern notions of sacred and profane,
and embody both \quoted{piety and play}.%
    \Autocite{Cashner:Cards}
If we perform villancicos with too much solemnity, listeners may miss the
elements of fun and virtuosity;
but if we perform them too flippantly, the audience may fail to recognize them
as expressions of human spirituality and ingenuity.

These pieces cannot be cleanly separated from the social values of the colonial
era that this music both reflected and reinforced.
All of the music in this edition was the product of an empire built on land
stolen from indigenous people, by the labor of enslaved men and women.
Please take note of the anti-Semitic and anti-Protestant slurs in the coplas
of Gutiérrez de Padilla's \term{jácara}, and consider omitting these verses in
public performance.%
    \Autocites
    [On Spanish anti-Semitism and representations of Jews, see][]
    {Martinez:GenealogicalFictions}
    {Nemiroff:ComediasJudaizantes}
Gutiérrez de Padilla was himself a slaveholder as well as an Oratorian priest,
and he perpetuates the typical prejudices of men of his social class.%
    \Autocite
    {Mauleon:PadillaCivil}
We owe our audiences better than to revive these notions or pretend that they
do not matter.%
    \Autocites
    {Baker:PerformancePostColonial}
    {Cashner:ImitatingAfricans}

\section{Acknowledgments}

This work was completed in Rochester, New York, on the traditional territory
of the Onöndowa'ga:' (Seneca) nation, one of the Six Nations of the
Haudenosaunee (Iroquois) Confederacy.
This edition is based on archival research in Puebla, Mexico City, Barcelona,
Madrid, and Segovia, in 2012.
This research was supported by a Jacob K. Javits Fellowship from the United
States Department of Education, an ACLS/Mellon Dissertation Completion
Fellowship, and travel grants from the University of Chicago Center for Latin
American Studies, Columbia University's Center for European Studies, and the
American Musicological Society.
The final stage was supported by a Humanities Center Fellowship from the
University of Rochester.

The edition was prepared using free and open-source software.
The music was typeset in Lilypond using an extensive custom library and the
text was typeset in \LaTeX{} using the editor's own packages,
\term{semantic-markup}, \term{octave}, \term{musicography} and
\term{poemtranslation}.
The complete sources for this edition are in a Git repository hosted on
Bitbucket at \url{http://www.bitbucket.com/andrewacashner/wlscm36}; the
Lilypond modules and LaTeX packages are in separate repositories on the same
site.
Please check the editor's website,
\url{http://www.andrewcashner.com/villancicos/}, for updates and corrections,
and please do not hesitate to report any errors.
Performing parts and transposed editions are available upon request.

\endinput




