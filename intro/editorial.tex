% editorial policies
% 2020/07/06

The edition preserves the original titles, section headings, and names for
voices and instruments, though I have standardized the labels indicating
choruses and voice numbers: \term{Tiple I-1} means Chorus I, first tiple
(treble).
In several pieces the bass parts contain only brief textual incipits because
they were meant to be played instrumentally, and some of them include explicit
instrument names.
I have enclosed all editorial section headings and other added text in square
brackets.
For sectional repeats I have substituted modern standard notation (repeat
barlines, \term{D.~S. al Fine}) or common-sense directions (\quoted{To
coplas}) for the various ways of indicating repetition in the sources.

The preparatory staves show the original clef, meter signature, and pitch,
using symbols close to those used in the manuscripts.
I traced the CZ symbol in Inkscape from the one used in the performing parts
of Miguel de Irízar from Segovia Cathedral.

In performing parts, the music was written in mensural notation with few
barlines except at section divisions.
In scores, composers did use barlines.
Miguel de Irízar placed barlines every two measures (\term{compases},
equivalent to \term{tactus}), just as the theorist Pedro Cerone recommended.%
    \Autocite[745]{Cerone:Melopeo}.
The most common meters are notated C and CZ.%
    \Autocites
    [537]{Cerone:Melopeo}
    [156, 165, 210]{Lorente:Porque}
    {GonzalezValle:CompasCabezon}
The first is a duple meter with two minim pulses per \term{compás}:
this is \term{compasillo} or \term{tiempo imperfecto menor}.
The second is a ternary meter with three minim pulses per \term{compás}; CZ
(or sometimes just Z) is a cursive shorthand for C3 or C\musFig{3 2},
\term{tiempo imperfecto menor de proporción menor}.
In ternary meter, coloration through blackening in noteheads indicated
imperfection (note values of two rather than three) and signaled a shift from
the default groupings of three minims.

In the edition I use \meterC{} and \meterCThree{} for duple and triple meter,
respectively.
I indicate mensural coloration with short square brackets.
I indicate ligatures with long brackets.
I add barlines every \term{compás} or metrical unit.
These are merely a convenience for modern performers, and do not indicate a
regular pattern of strong and weak beats (though in many cases such a pattern
does apply).
In translating the mensural notation to modern notation I have preserved the
original note values, so that a minim (half-note) in the original is still a
minim in the edition, though I have had to subdivide and tie notes across
barlines.
Think of \meterC{} as \musFig{2 2} and \meterCThree{} as \musFig{3 2}, performed
\quoted{in one}.

Beaming patterns are original.
In ternary meter, semiminims (quarter notes) are often written as flagged
minims, which are beamed together for melismas in the same way as
\term{corcheas} (\term{fusae}, eighth notes).
I have demarcated these groups with slurs.
Explicit slurs in the original, always indicated melismas and not phrasing,
are preserved, though I have standardized their start and end points.

All pieces are edited at their original notated pitch level.
As a practical aid I also provide a transposed edition of one piece whose
original tessitura would be an impediment to modern ensembles.
Accidentals are modernized but always in a way that preserves the meaning of
the original notation.
I have added suggestions for \term{musica ficta} accidental inflections above
the staff.
Figured bass is original.

I standardized and modernized the original phonetic orthography according to
modern Spanish except where doing so would alter the original sound.
For example, I preserve \term{acentos} instead of changing it to
\term{accentos}, which adds a hard C sound.
Where a symbol indicated repetitions of lyrical text, I supply the repeated
text in italic type.

\endinput




