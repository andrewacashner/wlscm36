\noteshead{Miguel Ambiela, \wtitle{Suban las voces al cielo}}
\begin{notesources}
    \begin{source}
        \sourcedescription{\signature{E-Bbc}{M733/1}, Manuscript performing
        parts}
        \annotation{\add{on title leaf:}
        Villancico a la Asunción/ de la Virgen a 6 Voces/
        dize/ Suban las vozes al cielo/
        Del Maestro Miguel Ambiela/
        que fue de Lerida y despues de daroca en donde el muro/
        es grande y la Ciutat es Poca/
        Torrente, Ente, Pente, mente, dente/
        Torrente;
        \add{on accompaniment part:}
        Acompañamiento Continuo a 6 Vozes del año 1689 a 24 octobre}
        \parts{SST, SAT, \term{Acompañamiento Continuo}}
    \end{source}

    \begin{source}
        \sourcedescription{Modern edition:
        \fullcite[35--45]{Calahorra:DarocaEdition}}
    \end{source}
\end{notesources}

Calahorra's edition does not indicate mensural coloration or editorial text.
In the fugato section, the Alto II that enters on \foreign{Vuelen, vuelen
juntas} in \measure{22} of that edition should enter on the fourth semiminim
\measure{23}.

As Ambiela asks for a repeat of a different portion of the estribillo after
each copla, I have opted to write out the repeated section fully to avoid
confusion.



\endinput
