\noteshead{Miguel de Irízar, \wtitle{Qué música divina}}
\begin{notesources}
    \begin{source}
        \sourcedescription{\signature{E-SE}{18/36}, Manuscript draft score in
        composer's hand}
        \annotation{Jesús. María. Y Joseph me ayuden. Fiesta del Nacimiento
        del año de 1678}
        \parts{SSAT, SATB, SATB, \term{General}}
    \end{source}
    \begin{source}
        \sourcedescription{\signature{E-SE}{3/2}, Manuscript performing parts}
        \annotation{Villancico para la Calenda al Nacimiento a 12, año de
        1678}
        \parts{SSAT, SATB, SATB, \term{General}}
    \end{source}
    \begin{source}
        \sourcedescription{\signature{E-Mn}{R/34982, VE/88/42}, two copies of
        poetry imprint from Toledo Cathedral, Christmas 1677}
    \end{source}
    \begin{source}
        \sourcedescription{Manuel de León Marchante, \wtitle{Obras poeticas
        posthumas} (Madrid, 1733), 169--170; later edition of Toledo 1677
        text}
    \end{source}
\end{notesources}

Irízar drafted this \term{villancico de calenda} on the empty space in the
margins and reverse sides of his received letters which he had fashioned into
a makeshift notebook.
Irízar began this piece with a prayer to the Holy Family on one of the center
openings in a sideways orientation and then flipped pages toward himself to
move toward what is now the front of the notebook, writing the music on the
reverse sides of his letters.
The manuscript includes a draft of one of the coplas which Irízar later
replaced with the version that appears in the corresponding parts.

While the performing parts were written without metrical barlines, Irízar did
use barlines to organize the score, usually in groups of two \term{compases}
or measures.
When a note was syncopated across the barline, since Irízar and his peers did
not use ties, Irízar just draws the blackened notehead centered on the
barline.

Irízar obtained this text, later attributed to Manuel de León Marchante, from
a poetry imprint from Toledo Cathedral the previous year, which Toledo
chapelmaster Pedro de Ardanaz, a fellow former pupil of Tomás Miciezes the
elder, had mailed to him.%
    \Autocite{Rodriguez:SoloMadrid}

This edition is based primarily on the parts, since they were actually used
for performance.
There are a few small differences in details of figured bass, mensural
coloration, accidentals, and text underlay.
In general, the parts, apparently professionally copied, are more precise and
consistent than Irízar's original draft.
The draft includes numerous corrections, especially of unintentional harmonic
clashes in the third-choir parts.

\endinput
