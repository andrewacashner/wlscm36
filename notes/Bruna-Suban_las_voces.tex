\noteshead{Pablo Bruna, \wtitle{Suban las voces al cielo}}
\begin{notesources}
    \begin{source}
        \sourcedescription{\signature{E-Bbc}{M759/44}, Manuscript performing
        parts, previously unattributed}
        \parts{SSAT}
    \end{source}
    \begin{source}
        \sourcedescription{\fullcite[29--35]{Calahorra:DarocaEdition}, based
        only on manuscript parts in Girona Cathedral Archive, no signature
        given}
        \parts{SSAT, \term{Entablatura}}
    \end{source}
\end{notesources}

Calahorra Martínez edited this work from a source in Girona that I have not
been able to locate.
That source includes a different poetic text for the coplas, which after the
first strophe, is in a meter that does not fit the musical setting.

The present edition is based solely on a separate source, a previously
unattributed set of parts in the Biblioteca de Catalunya, which agrees with
the Girona source in the vocal parts but has different copla text.
This text, I argue, is the original text of the coplas, and it is more
coherent poetically and theologically.
This version lacks the \term{entablatura} part that is found in the Girona
source; it is primarily a \term{basso seguente} that may have been added
later.

In the coplas, it would have been common practice for performers to adjust the
rhythm of the melody to fit the prosody of the subsequent strophes.
\endinput
