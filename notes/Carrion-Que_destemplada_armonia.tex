\noteshead{Jerónimo de Carrión, \wtitle{Qué destemplada armoníá}}
\begin{notesources}
    \begin{source}
        \sourcedescription{\signature{E-SE}{20/5}, manuscript performing
        parts}
        \annotation{Villancico de Calenda al Nacimiento de Nuestro Señor, a
        11}
        \parts{SSAT, SATB, SSB \term{chirimías} (\term{Tiple de chirimía},
        \term{Bajo de chirmiía}), \term{Acompañamiento de 1\textsuperscript{o}
        y 2\textsuperscript{o} coros}, \term{Acompañamiento de
        3\textsuperscript{o} coro al órgano}, \term{Acompañamiento general}}
    \end{source}
\end{notesources}

There are three accompaniment parts: one that supports Choir I and
II, another that supports Choir III, and a general continuo part.
The Choir III part specifies organ.

The bulk of the piece is carried by the soloists of Chorus I.
The third choir is a purely instrumental ensemble of \term{chirimías}
(shawms).

The dynamic and tempo markings for the different sections are a distinctive
feature of later seventeenth-century villancicos.
Also in line with contemporary trends, the estribillo has multiple sections
with contrasting textures and styles, and it is so long that Carrión only has
a \quoted{tag line} repeated after the coplas.

\endinput
