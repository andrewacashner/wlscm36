\noteshead{Juan Gutiérrez de Padilla, \wtitle{Afuera, afuera, pastores
(Jácara)}}
\begin{notesources}
    \begin{source}
        \sourcedescription{\signature{MEX-Pc}{Leg. 1/3}, in manuscript
        partbooks, \wtitle{Navidad de el año de 1652}}
        \annotation{Jácara a 6}
        \parts{SATB, B; Bassus I instrumental only, no text underlay; Bassus
        II instrumental and vocal with underlay}
    \end{source}
    \begin{source}
        \sourcedescription{Feliz Persio Bertiso, \wtitle{La harpa de Belen, en
        qve se cantan ocho Letrillas, y Chançonetas la noche de Nauidad, a el
        Santissimo Nacimiento de Christo Nuestro Señor. La primera es vn
        Romance muy ingenioso \Dots. La setima, a lo escarramanado de vn
        valenton que viene a visitar al Niño. \Dots} (Seville, 1634),
        poetry imprint in binder's collection,
        \signature{MEX-Plf}{80070-42010404-7}} 
        \annotation{Romance a lo valenton.}
    \end{source}
\end{notesources}

Gutiérrez de Padilla took this poem from a collection published to commemorate
a celebration at the Convento de Santa Inés in Seville 1634 (according to the
imprint's dedication). 
This imprint is preserved in a large binder's collection of other villancico
poems, chronicles of festivals, and related poetry and devotional literature,
much of it focused around Christmas and the Immaculate Conception of Mary.
There are numerous direct correspondences between texts in this collection and
villancicos set by Juan Gutiérrez de Padilla, and it is my belief that this
was originally his personal collection.
If not, the Puebla chapelmaster had his own set that included many of the same
imprints.

The poem plays on the jargon of outlaws and heroic warriors and is therefore
quite challenging to translate, and some of the meanings remain unclear to me.
I have not underlaid all fifteen strophes of the \term{romance}.
The performer likely would have been expected to adjust the melody and rhythm
slightly to accommodate the varied prosody of the subsequent coplas.

In Gutiérrez de Padilla's villancico cycles after 1652, the Bassus part of
both choirs includes only textual incipits, and there are numerous indications
throughout the partbooks that these voices were performed on the \term{bajón}
(dulcian, bass curtal; probably along with other continuo instruments like
harp and organ).
Here, though, the Bassus II has full text underlay and was clearly meant to be
sung, though it seems likely it was doubled on \term{bajón}.


\endinput
